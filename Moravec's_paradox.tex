

\documentclass[a4paper,10pt]{article} % Uses article class in A4 format

%----------------------------------------------------------------------------------------
%	FORMATTING
%----------------------------------------------------------------------------------------

\setlength{\parskip}{0pt}
\setlength{\parindent}{0pt}
\setlength{\voffset}{-15pt}

%----------------------------------------------------------------------------------------
%	PACKAGES AND OTHER DOCUMENT CONFIGURATIONS
%----------------------------------------------------------------------------------------

\usepackage[a4paper, margin=2.5cm]{geometry} % Sets margin to 2.5cm for A4 Paper
\usepackage[onehalfspacing]{setspace} % Sets Spacing to 1.5

\usepackage[T1]{fontenc} % Use European encoding
\usepackage[utf8]{inputenc} % Use UTF-8 encoding
\usepackage{charter} % Use the Charter font
\usepackage{microtype} % Slightly tweak font spacing for aesthetics

\usepackage[english, ngerman]{babel} % Language hyphenation and typographical rules

\usepackage{amsthm, amsmath, amssymb} % Mathematical typesetting
\usepackage{marvosym, wasysym} % More symbols
\usepackage{float} % Improved interface for floating objects
\usepackage[final, colorlinks = true, 
            linkcolor = black, 
            citecolor = black,
            urlcolor = black]{hyperref} % For hyperlinks in the PDF
\usepackage{graphicx, multicol} % Enhanced support for graphics
\usepackage{xcolor} % Driver-independent color extensions
\usepackage{rotating} % Rotation tools
\usepackage{listings, style/lstlisting} % Environment for non-formatted code, !uses style file!
\usepackage{pseudocode} % Environment for specifying algorithms in a natural way
\usepackage{style/avm} % Environment for f-structures, !uses style file!
\usepackage{booktabs} % Enhances quality of tables

\usepackage{tikz-qtree} % Easy tree drawing tool
\tikzset{every tree node/.style={align=center,anchor=north},
         level distance=2cm} % Configuration for q-trees
\usepackage{style/btree} % Configuration for b-trees and b+-trees, !uses style file!

\usepackage{titlesec} % Allows customization of titles
\renewcommand\thesection{\arabic{section}.} % Arabic numerals for the sections
\titleformat{\section}{\large}{\thesection}{1em}{}
\renewcommand\thesubsection{\alph{subsection})} % Alphabetic numerals for subsections
\titleformat{\subsection}{\large}{\thesubsection}{1em}{}
\renewcommand\thesubsubsection{\roman{subsubsection}.} % Roman numbering for subsubsections
\titleformat{\subsubsection}{\large}{\thesubsubsection}{1em}{}

\usepackage[all]{nowidow} % Removes widows

\usepackage[backend=biber,style=numeric,
            sorting=nyt, natbib=true]{biblatex} % Complete reimplementation of bibliographic facilities
\addbibresource{main.bib}
\usepackage{csquotes} % Context sensitive quotation facilities

\usepackage[yyyymmdd]{datetime} % Uses YEAR-MONTH-DAY format for dates
\renewcommand{\dateseparator}{-} % Sets dateseparator to '-'

\usepackage{fancyhdr} % Headers and footers
\pagestyle{fancy} % All pages have headers and footers
\fancyhead{}\renewcommand{\headrulewidth}{0pt} % Blank out the default header
\fancyfoot[L]{\textsc{}} % Custom footer text
\fancyfoot[C]{} % Custom footer text
\fancyfoot[R]{\thepage} % Custom footer text

\newcommand{\note}[1]{\marginpar{\scriptsize \textcolor{red}{#1}}} % Enables comments in red on margin

%----------------------------------------------------------------------------------------

\begin{document}

%----------------------------------------------------------------------------------------
%	TITLE SECTION
%----------------------------------------------------------------------------------------


\title{template_assignment} % Article title
\fancyhead[C]{}
\begin{minipage}{0.295\textwidth} % Left side of title section
\raggedright
AI and NN\\ % Your lecture or course
\footnotesize % Authors text size
%\hfill\\ % Uncomment if right minipage has more lines
Malaya Jhanwar, 19111033 % Your name, your matriculation number
\medskip\hrule
\end{minipage}
\begin{minipage}{0.4\textwidth} % Center of title section
\centering 
\large % Title text size
Moravec's paradox\\ % Assignment title and number
\normalsize % Subtitle text size
Assignment 3\\ % Assignment subtitle
\end{minipage}
\begin{minipage}{0.295\textwidth} % Right side of title section
\raggedleft
2021-07-19\\ % Date2021-07-19
\footnotesize % Email text size
%\hfill\\ % Uncomment if left minipage has more lines
jhanwarmalaya@gmail.com% Your email
\medskip\hrule
\end{minipage}

%----------------------------------------------------------------------------------------
%	ARTICLE CONTENTS
%----------------------------------------------------------------------------------------

% here be dragons

\section{Introduction}

Moravec’s paradox is a phenomenon surrounding the abilities of AI-powered tools. It observes that tasks humans find complex are easy to teach AI. Compared, that is, to simple, sensorimotor skills that come instinctively to humans.

In the 1980s, Hans Moravec, Rodney Brooks, Marvin Minsky and others articulated and discussed this AI paradox. As Moravec put it:\\
\textbf{“It is comparatively easy to make computers exhibit adult level performance  and difficult or impossible to give them the skills of a one-year-old.”}

\section{Logic behind Moravec’s paradox}
For a start, the skills that we define as ‘simple’ — those we learn instinctively — are products of years and years of evolution. So, while they may appear simple, it’s only because of billions of years’ worth of tuning.

Plus, AI ‘learns’ through us telling it how to do things. We’ve consciously learned how to do mathematics, win games and follow logic. We know the steps (computations) needed to complete these tasks. And so, we can teach them to AI.

But how do you tell anything how to see, hear, or move?

We don’t consciously know all the computations needed to complete these tasks. These skills are not broken down into logical steps to feed into an AI. As such, teaching them to an AI is extremely difficult

\section{Biological basis of human skills}

As Moravec writes:

Encoded in the large, highly evolved sensory and motor portions of the human brain is a billion years of experience about the nature of the world and how to survive in it. Abstract thought, though, is a new trick, perhaps less than 100 thousand years old. We have not yet mastered it. It is not all that intrinsically difficult; it just seems so when we do it.

A compact way to express this argument would be:
\begin{itemize}
    \item We should expect the difficulty of reverse-engineering any human skill to be roughly proportional to the amount of time that skill has been evolving in animals.
    \item The oldest human skills are largely unconscious and so appear to us to be effortless.
    \item Therefore, we should expect skills that appear effortless to be difficult to reverse-engineer, but skills that require effort may not necessarily be difficult to engineer at all.
\end{itemize}



\section{Conclusion}
It’s not easy to interpret Moravec’s paradox. Some tell that it describes the future where machines will take jobs which require specialistic skills, making people serving an army of robotic chiefs and analysts. Others argue that paradox guarantees that AI will always need an assistance of people. Or, perhaps more correctly, people will use AI to improve those skills which aren’t as highly developed by nature.
For sure Moravec’s paradox proves one thing — the fact that we developed computer to beat human in Go or Chess doesn’t mean that General Artificial Intelligence is just around the corner. Yes, we are one step closer. But as long as AGI means for us “full copy of human intelligence”, over time it will be only harder.


\end{document}
