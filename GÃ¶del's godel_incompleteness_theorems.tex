
\documentclass[a4paper,10pt]{article} % Uses article class in A4 format



\setlength{\parskip}{0pt}
\setlength{\parindent}{0pt}
\setlength{\voffset}{-15pt}


\usepackage[a4paper, margin=2.5cm]{geometry} % Sets margin to 2.5cm for A4 Paper
\usepackage[onehalfspacing]{setspace} % Sets Spacing to 1.5

\usepackage[T1]{fontenc} % Use European encoding
\usepackage[utf8]{inputenc} % Use UTF-8 encoding
\usepackage{charter} % Use the Charter font
\usepackage{microtype} % Slightly tweak font spacing for aesthetics

\usepackage[english, ngerman]{babel} % Language hyphenation and typographical rules

\usepackage{amsthm, amsmath, amssymb} % Mathematical typesetting
\usepackage{marvosym, wasysym} % More symbols
\usepackage{float} % Improved interface for floating objects
\usepackage[final, colorlinks = true, 
            linkcolor = black, 
            citecolor = black,
            urlcolor = black]{hyperref} % For hyperlinks in the PDF
\usepackage{graphicx, multicol} % Enhanced support for graphics
\usepackage{xcolor} % Driver-independent color extensions
\usepackage{rotating} % Rotation tools
\usepackage{listings, style/lstlisting} % Environment for non-formatted code, !uses style file!
\usepackage{pseudocode} % Environment for specifying algorithms in a natural way
\usepackage{style/avm} % Environment for f-structures, !uses style file!
\usepackage{booktabs} % Enhances quality of tables

\usepackage{tikz-qtree} % Easy tree drawing tool
\tikzset{every tree node/.style={align=center,anchor=north},
         level distance=2cm} % Configuration for q-trees
\usepackage{style/btree} % Configuration for b-trees and b+-trees, !uses style file!

\usepackage{titlesec} % Allows customization of titles
\renewcommand\thesection{\arabic{section}.} % Arabic numerals for the sections
\titleformat{\section}{\large}{\thesection}{1em}{}
\renewcommand\thesubsection{\alph{subsection})} % Alphabetic numerals for subsections
\titleformat{\subsection}{\large}{\thesubsection}{1em}{}
\renewcommand\thesubsubsection{\roman{subsubsection}.} % Roman numbering for subsubsections
\titleformat{\subsubsection}{\large}{\thesubsubsection}{1em}{}

\usepackage[all]{nowidow} % Removes widows

\usepackage[backend=biber,style=numeric,
            sorting=nyt, natbib=true]{biblatex} % Complete reimplementation of bibliographic facilities
\addbibresource{main.bib}
\usepackage{csquotes} % Context sensitive quotation facilities

\usepackage[yyyymmdd]{datetime} % Uses YEAR-MONTH-DAY format for dates
\renewcommand{\dateseparator}{-} % Sets dateseparator to '-'

\usepackage{fancyhdr} % Headers and footers
\pagestyle{fancy} % All pages have headers and footers
\fancyhead{}\renewcommand{\headrulewidth}{0pt} % Blank out the default header
\fancyfoot[L]{\textsc{}} % Custom footer text
\fancyfoot[C]{} % Custom footer text
\fancyfoot[R]{\thepage} % Custom footer text

\newcommand{\note}[1]{\marginpar{\scriptsize \textcolor{red}{#1}}} % Enables comments in red on margin

%----------------------------------------------------------------------------------------

\begin{document}

%----------------------------------------------------------------------------------------
%	TITLE SECTION
%----------------------------------------------------------------------------------------

\title{template_assignment} % Article title
\fancyhead[C]{}
\begin{minipage}{0.295\textwidth} % Left side of title section
\raggedright
AI and NN\\ % Your lecture or course
\footnotesize % Authors text size
%\hfill\\ % Uncomment if right minipage has more lines
Malaya Jhanwar, 19111033 % Your name, your matriculation number
\medskip\hrule
\end{minipage}
\begin{minipage}{0.4\textwidth} % Center of title section
\centering 
\large % Title text size
Gödel's incompleteness theorems\\ % Assignment title and number
\normalsize % Subtitle text size
Assignment 2\\ % Assignment subtitle
\end{minipage}
\begin{minipage}{0.295\textwidth} % Right side of title section
\raggedleft
\today\\ % Date
\footnotesize % Email text size
%\hfill\\ % Uncomment if left minipage has more lines
jhanwarmalaya@gmail.com% Your email
\medskip\hrule
\end{minipage}

\section{Introduction}
Gödel's incompleteness theorems are two theorems of mathematical logic that are concerned with the limits of provability in formal axiomatic theories. These results, published by Kurt Gödel in 1931, are important both in mathematical logic and in the philosophy of mathematics. The theorems are widely, but not universally, interpreted as showing that Hilbert's program to find a complete and consistent set of axioms for all mathematics is impossible.

The first incompleteness theorem states that no consistent system of axioms whose theorems can be listed by an effective procedure (i.e., an algorithm) is capable of proving all truths about the arithmetic of natural numbers. For any such consistent formal system, there will always be statements about natural numbers that are true, but that are unprovable within the system. The second incompleteness theorem, an extension of the first, shows that the system cannot demonstrate its own consistency.

Employing a diagonal argument, Gödel's incompleteness theorems were the first of several closely related theorems on the limitations of formal systems. They were followed by Tarski's undefinability theorem on the formal undefinability of truth, Church's proof that Hilbert's Entscheidungsproblem is unsolvable, and Turing's theorem that there is no algorithm to solve the halting problem.

\section{Incompleteness theorems}
\subsection{First Theorem}
 "Any consistent formal system F within which a certain amount of elementary arithmetic can be carried out is incomplete; i.e., there are statements of the language of F which can neither be proved nor disproved in F."
 
 The unprovable statement GF referred to by the theorem is often referred to as "the Gödel sentence" for the system F. The proof constructs a particular Gödel sentence for the system F, but there are infinitely many statements in the language of the system that share the same properties, such as the conjunction of the Gödel sentence and any logically valid sentence.

Each effectively generated system has its own Gödel sentence. It is possible to define a larger system F’ that contains the whole of F plus GF as an additional axiom. This will not result in a complete system, because Gödel's theorem will also apply to F’, and thus F’ also cannot be complete. In this case, GF is indeed a theorem in F’, because it is an axiom.
 
 \subsection{Second Theorem}
 "Assume F is a consistent formalized system which contains elementary arithmetic. Then {\displaystyle F\not \vdash {\text{Cons}}(F)}{\displaystyle F\not \vdash {\text{Cons}}(F)}."
 
 Gödel's second incompleteness theorem shows that, under general assumptions, this canonical consistency statement Cons(F) will not be provable in F. The theorem first appeared as "Theorem XI" in Gödel's 1931 paper "On Formally Undecidable Propositions in Principia Mathematica and Related Systems I". In the following statement, the term "formalized system" also includes an assumption that F is effectively axiomatized.

This theorem is stronger than the first incompleteness theorem because the statement constructed in the first incompleteness theorem does not directly express the consistency of the system. The proof of the second incompleteness theorem is obtained by formalizing the proof of the first incompleteness theorem within the system F itself.
 
 
 
 

\end{document}
