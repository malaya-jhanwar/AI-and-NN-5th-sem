%%%%%%%%%%%%%%%%%%%%%%%%%%%%%%%%%%%%%%%%%
% Homework Assignment Article
% LaTeX Template
% Version 1.3.5r (2018-02-16)
%
% This template has been downloaded from:
% /cl.uni-heidelberg.de/~zimmermann/
%
% Original author:
% Victor Zimmermann (zimmermann@cl.uni-heidelberg.de)
%
% License:
% CC BY-SA 4.0 (https://creativecommons.org/licenses/by-sa/4.0/)
%
%%%%%%%%%%%%%%%%%%%%%%%%%%%%%%%%%%%%%%%%%

%----------------------------------------------------------------------------------------

\documentclass[a4paper,10pt]{article} % Uses article class in A4 format

%----------------------------------------------------------------------------------------
%	FORMATTING
%----------------------------------------------------------------------------------------

\setlength{\parskip}{0pt}
\setlength{\parindent}{0pt}
\setlength{\voffset}{-15pt}

%----------------------------------------------------------------------------------------
%	PACKAGES AND OTHER DOCUMENT CONFIGURATIONS
%----------------------------------------------------------------------------------------

\usepackage[a4paper, margin=2.5cm]{geometry} % Sets margin to 2.5cm for A4 Paper
\usepackage[onehalfspacing]{setspace} % Sets Spacing to 1.5

\usepackage[T1]{fontenc} % Use European encoding
\usepackage[utf8]{inputenc} % Use UTF-8 encoding
\usepackage{charter} % Use the Charter font
\usepackage{microtype} % Slightly tweak font spacing for aesthetics

\usepackage[english, ngerman]{babel} % Language hyphenation and typographical rules

\usepackage{amsthm, amsmath, amssymb} % Mathematical typesetting
\usepackage{marvosym, wasysym} % More symbols
\usepackage{float} % Improved interface for floating objects
\usepackage[final, colorlinks = true, 
            linkcolor = black, 
            citecolor = black,
            urlcolor = black]{hyperref} % For hyperlinks in the PDF
\usepackage{graphicx, multicol} % Enhanced support for graphics
\usepackage{xcolor} % Driver-independent color extensions
\usepackage{rotating} % Rotation tools
\usepackage{listings, style/lstlisting} % Environment for non-formatted code, !uses style file!
\usepackage{pseudocode} % Environment for specifying algorithms in a natural way
\usepackage{style/avm} % Environment for f-structures, !uses style file!
\usepackage{booktabs} % Enhances quality of tables

\usepackage{tikz-qtree} % Easy tree drawing tool
\tikzset{every tree node/.style={align=center,anchor=north},
         level distance=2cm} % Configuration for q-trees
\usepackage{style/btree} % Configuration for b-trees and b+-trees, !uses style file!

\usepackage{titlesec} % Allows customization of titles
\renewcommand\thesection{\arabic{section}.} % Arabic numerals for the sections
\titleformat{\section}{\large}{\thesection}{1em}{}
\renewcommand\thesubsection{\alph{subsection})} % Alphabetic numerals for subsections
\titleformat{\subsection}{\large}{\thesubsection}{1em}{}
\renewcommand\thesubsubsection{\roman{subsubsection}.} % Roman numbering for subsubsections
\titleformat{\subsubsection}{\large}{\thesubsubsection}{1em}{}

\usepackage[all]{nowidow} % Removes widows

\usepackage[backend=biber,style=numeric,
            sorting=nyt, natbib=true]{biblatex} % Complete reimplementation of bibliographic facilities
\addbibresource{main.bib}
\usepackage{csquotes} % Context sensitive quotation facilities

\usepackage[yyyymmdd]{datetime} % Uses YEAR-MONTH-DAY format for dates
\renewcommand{\dateseparator}{-} % Sets dateseparator to '-'

\usepackage{fancyhdr} % Headers and footers
\pagestyle{fancy} % All pages have headers and footers
\fancyhead{}\renewcommand{\headrulewidth}{0pt} % Blank out the default header
\fancyfoot[L]{\textsc{ 00}} % Custom footer text
\fancyfoot[C]{} % Custom footer text
\fancyfoot[R]{\thepage} % Custom footer text

\newcommand{\note}[1]{\marginpar{\scriptsize \textcolor{red}{#1}}} % Enables comments in red on margin

%----------------------------------------------------------------------------------------

\begin{document}

%----------------------------------------------------------------------------------------
%	TITLE SECTION
%----------------------------------------------------------------------------------------

\title{template_assignment} % Article title
\fancyhead[C]{}
\begin{minipage}{0.295\textwidth} % Left side of title section
\raggedright
AI and NN\\ % Your lecture or course
\footnotesize % Authors text size
%\hfill\\ % Uncomment if right minipage has more lines
Malaya Jhanwar, 19111033 % Your name, your matriculation number
\medskip\hrule
\end{minipage}
\begin{minipage}{0.4\textwidth} % Center of title section
\centering 
\large % Title text size
Assignment 01\\ % Assignment title and number

\end{minipage}
\begin{minipage}{0.295\textwidth} % Right side of title section
\raggedleft
\today\\ % Date
\footnotesize % Email text size
%\hfill\\ % Uncomment if left minipage has more lines
jhanwarmalaya@gmail.com% Your email
\medskip\hrule
\end{minipage}

%----------------------------------------------------------------------------------------
%	ARTICLE CONTENTS
%----------------------------------------------------------------------------------------

% here be dragons




\section{Philosophy Of Artificial Intelligence}

It is a branch of the philosophy of technology that explores artificial intelligence and its implications for knowledge and understanding of intelligence, ethics, consciousness, epistemology, and free will.
 The technology is concerned with the creation of artificial animals or artificial people (or, at least, artificial creatures; see artificial life) so the discipline is of considerable interest to philosophers.
 
 
 Philosophy of AI 
 explores artificial intelligence and its implications for “knowledge” and  understanding of intelligence, ethics, consciousness, epistemology, and free  will 


• The philosophy of artificial intelligence attempts to answer such questions as  follows

• Can a machine have a mind, mental states, and consciousness in the same sense that a  human being can? Can it feel how things are? 


 "\textbf{ Every aspect of learning or any other feature of intelligence can be so precisely described that a machine can be made to simulate it."}
  
  
  \section{Intelligence}

     \textbf{Turing test : If a machine acts as intelligently as a human being, then it is as intelligent as a human being.}
       \textbf{"If an agent acts so as to maximize the expected value of a performance measure based on past experience and knowledge then it is intelligent."}
  
 
 
 
 \section{Does a machine have a mind, consciousness, and mental states}
 The words "mind" and "consciousness" are used by different communities in different ways. Some new age thinkers, for example, use the word "consciousness" to describe something similar to Bergson's "élan vital": an invisible, energetic fluid that permeates life and especially the mind. Science fiction writers use the word to describe some essential property that makes us human: a machine or alien that is "conscious" will be presented as a fully human character, with intelligence, desires, will, insight, pride
 
 Neurobiologists believe all these problems will be solved as we begin to identify the neural correlates of consciousness: the actual relationship between the machinery in our heads and its collective properties; such as the mind, experience and understanding. 

\section{Chinese Room}Responses to the Chinese room emphasize several different points:

\begin{itemize}
    \item The systems reply and the virtual mind reply: This reply argues that the system, including the man, the program, the room, and the cards, is what understands Chinese. Searle claims that the man in the room is the only thing which could possibly "have a mind" or "understand", but others disagree, arguing that it is possible for there to be two minds in the same physical place, similar to the way a computer can simultaneously "be" two machines at once: one physical (like a Macintosh) and one "virtual" (like a word processor).
  
    \item Brain simulator reply: What if the program simulates the sequence of nerve firings at the synapses of an actual brain of an actual Chinese speaker? The man in the room would be simulating an actual brain. This is a variation on the "systems reply" that appears more plausible because "the system" now clearly operates like a human brain, which strengthens the intuition that there is something besides the man in the room that could understand Chinese.
    \item Other minds reply and the epiphenomena reply: Several people have noted that Searle's argument is just a version of the problem of other minds, applied to machines. Since it is difficult to decide if people are "actually" thinking, we should not be surprised that it is difficult to answer the same question about machines.
\end{itemize}

\section{Computational Theory}
The computational theory of mind or "computationalism"claims that the relationship between mind and
brain is similar (if not identical) to the relationship between a running program and a computer. The idea
has philosophical roots in Hobbes (who claimed reasoning was "nothing more than reckoning"), Leibniz
(who attempted to create a logical calculus of all human ideas), Hume (who thought perception could
be reduced to ätomic impressions") and even Kant (who analyzed all experience as controlled by formal
rules).[66] The latest version is associated with philosophers Hilary Putnam and Jerry Fodor.
This question bears on our earlier questions: if the human brain is a kind of computer then computers
can be both intelligent and conscious, answering both the practical and philosophical questions of AI. In
terms of the practical question of AI ("Can a machine display general intelligence?"), some versions of
computationalism make the claim that (as Hobbes wrote):


• Reasoning is nothing but reckoning


• Mental states are just implementations of computer programs.


 

\section{Machines With Emotions}
If "emotions" are defined only in terms of their effect on behavior or on how they function inside an organism, then emotions can be viewed as a mechanism that an intelligent agent uses to maximize the utility of its actions. Given this definition of emotion, Hans Moravec believes that "robots in general will be quite emotional about being nice people". Fear is a source of urgency. Empathy is a necessary component of good human computer interaction. He says robots "will try to please you in an apparently selfless manner because it will get a thrill out of this positive reinforcement. You can interpret this as a kind of love." Daniel Crevier writes "Moravec's point is that emotions are just devices for channeling behavior in a direction beneficial to the survival of one's species."

%----------------------------------------------------------------------------------------
%	REFERENCE LIST
%----------------------------------------------------------------------------------------


%----------------------------------------------------------------------------------------

\end{document}
